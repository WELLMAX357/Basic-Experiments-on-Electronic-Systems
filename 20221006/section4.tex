% !TEX root = main.tex

%%%%%%%%%%%%%%%%%%%%%%%%%%%%%%%%%%%%%%%%%%%%%%%%%%%%%%%%%%%%%%%%%%%%%%%%%%%%%%%%%%%%%%%%%%%%%%%%
\section{考察}
%%%%%%%%%%%%%%%%%%%%%%%%%%%%%%%%%%%%%%%%%%%%%%%%%%%%%%%%%%%%%%%%%%%%%%%%%%%%%%%%%%%%%%%%%%%%%%%%
\subsection{理論値との比較}
被測定回路$1,2,3$の4端子定数 $(A, B, C, D)$の測定値を理論値と比較して議論する.
このためには,各測定回路でのF行列の理論式を求め,その値を計算する必要がある.
\subsubsection{理論値}
    \begin{enumerate}
        \item 被測定回路1
        $$
        A_1=\frac{V_1}{V_2}=\frac{V_1}{\frac{Z_2}{Z_1+Z_2}V_1}=1+\frac{Z_1}{Z_2}=1+j\omega RC=1+j6.28=6.36\angle80.95^\circ
        $$
        $$
        B_1=\frac{V_1}{I_2}=\frac{V_1}{\frac{V_1}{Z_1}}=Z1=R=1.00\times10^3=1.00\times10^3\angle0.00^\circ
        $$
        $$
        C_1=\frac{I_1}{V_2}=\frac{I_1}{Z_2I_1}=\frac{1}{Z_2}=j\omega C=j6.28\times10^{-3}=6.28\times10^{-3}\angle90.00^\circ
        $$
        $$
        D_1=\frac{I_1}{I_2}=1=1.00\angle0.00^\circ
        $$
        \item 被測定回路2
        
        被測定回路1の入出力を入れ替えたものであるので
        $$
        A_2=1.00\angle0.00^\circ
        $$
        $$
        B_2=1.00\times10^3\angle0.00^\circ
        $$
        $$
        C_2=6.28\times10^{-3}\angle90.00^\circ
        $$
        $$
        D_2=6.36\angle80.95^\circ
        $$
        \item 被測定回路3
        
        被測定回路1と被測定回路2の縦続接続であるので
        $$
        \left[\begin{array}{ll}
        A_3 & B_3 \\
        C_3 & D_3
        \end{array}\right]=\left[\begin{array}{ll}
        A_1 & B_1 \\
        C_1 & D_1
        \end{array}\right]\left[\begin{array}{ll}
        A_2 & B_2 \\
        C_2 & D_2
        \end{array}\right]=\left[\begin{array}{ll}
        A_1A_2+B_1C_2 & A_1B_2+B_1D_2 \\
        C_1A_2+D_1C_2 & C_1B_2+D_1D_2
        \end{array}\right]
        $$
        $$
        =\left[\begin{array}{ll}
        12.60\angle85.45^\circ & 12.72\times10^3\angle80.95^\circ \\
        12.56\times10^{-3}\angle90^\circ & 12.60\angle85.45^\circ
        \end{array}\right]
        $$
    \end{enumerate}

\newpage

\subsubsection{測定値}
    \begin{enumerate}
        \item 被測定回路1
        \begin{itemize}
            \item 4端子定数$A,C$の測定
            $$
            |V_1|=8.20V
            $$
            $$
            |V_2|=1.15V
            $$
            $$
            |V_{10}|=80.00mV
            $$
            $$
            \theta_A=85.50^\circ
            $$
            $$
            \theta_C=82.80^\circ
            $$

            $$
            |A|=\frac{\left|V_1\right|}{\left|V_2\right|}=7.13
            $$
            $$
            |C|=\frac{\frac{\left|V_{10}\right|}{R_{10}}}{\left|V_2\right|}=\frac{\left|V_{10}\right|}{\left|V_2\right| R_{10}}=6.96\times10^{-3}
            $$

            \item 4端子定数$B,D$の測定
            $$
            |V_1|=7.50V
            $$
            $$
            |V_{10}|=80.00mV
            $$
            $$
            |V_{20}|=78.00mV
            $$
            $$
            \theta_B=0.00^\circ
            $$
            $$
            \theta_D=0.00^\circ
            $$

            $$
            |B|=\frac{\left|V_1\right|}{\frac{\left|V_{20}\right|}{R_{20}}}=\frac{\left|V_1\right| R_{20}}{\left|V_{20}\right|}=961.54
            $$
            $$
            |D|=\frac{\frac{\left|V_{10}\right|}{R_{10}}}{\frac{\left|V_{20}\right|}{R_{20}}}=\frac{\left|V_{10}\right| R_{20}}{\left|V_{20}\right| R_{10}}=1.03
            $$

        \end{itemize}

        \newpage
        
        \item 被測定回路2
        \begin{itemize}
            \item 4端子定数$A,C$の測定
            $$
            |V_1|=3.00V
            $$
            $$
            |V_2|=3.00V
            $$
            $$
            |V_{10}|=190.00mV
            $$
            $$
            \theta_A=0.00^\circ
            $$
            $$
            \theta_C=90.00^\circ
            $$

            $$
            |A|=\frac{\left|V_1\right|}{\left|V_2\right|}=1.00
            $$
            $$
            |C|=\frac{\frac{\left|V_{10}\right|}{R_{10}}}{\left|V_2\right|}=\frac{\left|V_{10}\right|}{\left|V_2\right| R_{10}}=6.33\times10^{-3}
            $$

            \item 4端子定数$B,D$の測定
            $$
            |V_1|=2.70V
            $$
            $$
            |V_{10}|=185.00mV
            $$
            $$
            |V_{20}|=34.00mV
            $$
            $$
            \theta_B=0.00^\circ
            $$
            $$
            \theta_D=86.40^\circ
            $$

            $$
            |B|=\frac{\left|V_1\right|}{\frac{\left|V_{20}\right|}{R_{20}}}=\frac{\left|V_1\right| R_{20}}{\left|V_{20}\right|}=794.12
            $$
            $$
            |D|=\frac{\frac{\left|V_{10}\right|}{R_{10}}}{\frac{\left|V_{20}\right|}{R_{20}}}=\frac{\left|V_{10}\right| R_{20}}{\left|V_{20}\right| R_{10}}=5.44
            $$
        \end{itemize}

        \newpage
        
        \item 被測定回路3
        \begin{itemize}
            \item 4端子定数$A,C$の測定
            $$
            |V_1|=7.20V
            $$
            $$
            |V_2|=620.00mV
            $$
            $$
            |V_{10}|=80.00mV
            $$
            $$
            \theta_A=86.40^\circ
            $$
            $$
            \theta_C=82.80^\circ
            $$

            $$
            |A|=\frac{\left|V_1\right|}{\left|V_2\right|}=11.61
            $$
            $$
            |C|=\frac{\frac{\left|V_{10}\right|}{R_{10}}}{\left|V_2\right|}=\frac{\left|V_{10}\right|}{\left|V_2\right| R_{10}}=12.90\times10^{-3}
            $$

            \item 4端子定数$B,D$の測定
            $$
            |V_1|=7.40V
            $$
            $$
            |V_{10}|=80.00mV
            $$
            $$
            |V_{20}|=14.00mV
            $$
            $$
            \theta_B=90.00^\circ
            $$
            $$
            \theta_D=90.00^\circ
            $$

            $$
            |B|=\frac{\left|V_1\right|}{\frac{\left|V_{20}\right|}{R_{20}}}=\frac{\left|V_1\right| R_{20}}{\left|V_{20}\right|}=5.29\times10^3
            $$
            $$
            |D|=\frac{\frac{\left|V_{10}\right|}{R_{10}}}{\frac{\left|V_{20}\right|}{R_{20}}}=\frac{\left|V_{10}\right| R_{20}}{\left|V_{20}\right| R_{10}}=5.71
            $$
        \end{itemize}
    \end{enumerate}

\newpage



\subsection{入出力端子を入れ替えた場合の関係}
被測定回路$1,2$の4端子定数の測定値より,入力端子と出力端子を入れ換えた場合の関係式
$$
A_2=D_1, B_2=B_1, C_2=C_1, D_2=A_1
$$
が成り立っているかどうかを検討する.

$$A_2=D_1$$
$$1.00=1.03$$

$$B_2=B_1$$
$$794.12=961.54$$

$$C_2=C_1$$
$$6.33\times10^{-3}=6.96\times10^{-3}$$

$$D_2=A_1$$
$$5.44=7.13$$

\subsection{自然回路としての関係式}
被測定回路$1,2,3$の4端子定数$(A, B, C, D)$の測定値より
$$
A D-B C=1
$$
が成り立っているかどうか検討する. 

\begin{enumerate}
    \item 被測定回路1
    $$
    A_1D_1-B_1C_1=0.65
    $$
    
    \item 被測定回路2
    $$
    A_2D_2-B_2C_2=0.41
    $$
    
    \item 被測定回路3
    $$
    A_3D_3-B_3C_3=-1.95
    $$
    
\end{enumerate}

\subsection{縦続接続}
被測定回路3は被測定回路$1,2$を縦続接続したものである.縦続接続の場合の関係式
$$
\left[\begin{array}{ll}
A_3 & B_3 \\
C_3 & D_3
\end{array}\right]=\left[\begin{array}{ll}
A_1 & B_1 \\
C_1 & D_1
\end{array}\right]\left[\begin{array}{cc}
A_2 & B_2 \\
C_2 & D_2
\end{array}\right]
$$
が成り立っているかどうか検討する.

$$
A_3=A_1A_2+B_1C_2=13.22
$$
$$
B_3=A_1B_2+B_1D_2=10.89\times10^3
$$
$$
C_3=C_1A_2+D_1C_2=13.48\times10^{-3}
$$
$$
D_3=C_1B_2+D_1D_2=11.13
$$


被測定回路1,2の測定値から,縦続接続の場合の関係式を用いた被測定回路3のF行列は理論とほぼ一致した.
このことから被測定回路1,2の測定は自然回路としての関係式で誤差が出たものの,測定方法に問題はなかったといえる.
被測定回路1,2において誤差が発生した原因は,目視によるオシロスコープ波形の測定方法に問題があったと考えられる.
波形に現れるノイズを小さくすることに意識をとらわれ,すべての実験においてRC発信機の出力を最大にしていた.
オシロスコープに表示された波形は画面いっぱいに表示されたものもあれば,画面よりも大きい波形が表示されたために,
VOLTS/DIVの値を大きくする必要があり,波形が画面の半分までしか表示されないことがあった.
波形が最も大きな状態で測定することができなかったために,目視による測定で誤差が生じたと考えられる.
各々の測定においてRC発信機の出力を最適にすべきであった.

一方被測定回路3の4端子定数$B,D$の測定は他の測定と比較して大きく誤差が出た.
測定箇所,または測定回路に問題があったと考えられる.他の測定と比較して使用した素子が多くなっているので,
誤った抵抗を回路に組み込んだ可能性がある.測定時には他の回路よりも小さな電圧値が出力され,ノイズが多く乗っていたため,
誤差がより大きくなったと考えられる.
