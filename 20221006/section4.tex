% !TEX root = main.tex

%%%%%%%%%%%%%%%%%%%%%%%%%%%%%%%%%%%%%%%%%%%%%%%%%%%%%%%%%%%%%%%%%%%%%%%%%%%%%%%%%%%%%%%%%%%%%%%%
\section{考察}
%%%%%%%%%%%%%%%%%%%%%%%%%%%%%%%%%%%%%%%%%%%%%%%%%%%%%%%%%%%%%%%%%%%%%%%%%%%%%%%%%%%%%%%%%%%%%%%%
\subsection{理論値との比較}
被測定回路$1,2,3$の4端子定数 $(A, B, C, D)$の測定値を理論値と比較して議論する.
このためには,各測定回路でのF行列の理論式を求め,その値を計算する必要がある.

\subsection{入出力端子を入れ替えた場合の関係}
被測定回路$1,2$の4端子定数の測定値より,入力端子と出力端子を入れ換えた場合の関係式
$$
A_2=D_1, B_2=B_1, C_2=C_1, D_2=A_1
$$
が成り立っているかどうかを検討する.

\subsection{自然回路としての関係式}
被測定回路$1,2,3$の4端子定数$(A, B, C, D)$の測定値より
$$
A D-B C=1
$$
が成り立っているかどうか検討する.  

\subsection{縦続接続}
被測定回路3は被測定回路$1,2$を縦続接続したものである.縦続接続の場合の関係式
$$
\left[\begin{array}{ll}
A_3 & B_3 \\
C_3 & D_3
\end{array}\right]=\left[\begin{array}{ll}
A_1 & B_1 \\
C_1 & D_1
\end{array}\right]\left[\begin{array}{cc}
A_2 & B_2 \\
C_2 & D_2
\end{array}\right]
$$
が成り立っているかどうか検討する.