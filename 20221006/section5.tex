% !TEX root = main.tex

%%%%%%%%%%%%%%%%%%%%%%%%%%%%%%%%%%%%%%%%%%%%%%%%%%%%%%%%%%%%%%%%%%%%%%%%%%%%%%%%%%%%%%%%%%%%%%%%
\section{考察}
%%%%%%%%%%%%%%%%%%%%%%%%%%%%%%%%%%%%%%%%%%%%%%%%%%%%%%%%%%%%%%%%%%%%%%%%%%%%%%%%%%%%%%%%%%%%%%%%
\subsection{理論値との比較}
被測定回路$1,2,3$の4端子定数 $(A, B, C, D)$の測定値を理論値と比較して議論する.
このためには,各測定回路でのF行列の理論式を求め,その値を計算する必要がある.
\subsection*{理論値}
    \begin{description}
        \item 被測定回路1
        $$
        A_1=\frac{V_1}{V_2}=\frac{V_1}{\frac{Z_2}{Z_1+Z_2}V_1}=1+\frac{Z_1}{Z_2}=1+j\omega RC=1+j6.28=6.36\angle80.95^\circ
        $$
        $$
        B_1=\frac{V_1}{I_2}=\frac{V_1}{\frac{V_1}{Z_1}}=Z1=R=1.00\times10^3=1.00\times10^3\angle0.00^\circ
        $$
        $$
        C_1=\frac{I_1}{V_2}=\frac{I_1}{Z_2I_1}=\frac{1}{Z_2}=j\omega C=j6.28\times10^{-3}=6.28\times10^{-3}\angle90.00^\circ
        $$
        $$
        D_1=\frac{I_1}{I_2}=1=1.00\angle0.00^\circ
        $$
        \item 被測定回路2
        
        被測定回路1の入出力を入れ替えたものであるので
        $$
        A_2=1.00\angle0.00^\circ
        $$
        $$
        B_2=1.00\times10^3\angle0.00^\circ
        $$
        $$
        C_2=6.28\times10^{-3}\angle90.00^\circ
        $$
        $$
        D_2=6.36\angle80.95^\circ
        $$
        \item 被測定回路3
        
        被測定回路1と被測定回路2の縦続接続であるので
        $$
        \left[\begin{array}{ll}
        A_3 & B_3 \\
        C_3 & D_3
        \end{array}\right]=\left[\begin{array}{ll}
        A_1 & B_1 \\
        C_1 & D_1
        \end{array}\right]\left[\begin{array}{ll}
        A_2 & B_2 \\
        C_2 & D_2
        \end{array}\right]=\left[\begin{array}{ll}
        A_1A_2+B_1C_2 & A_1B_2+B_1D_2 \\
        C_1A_2+D_1C_2 & C_1B_2+D_1D_2
        \end{array}\right]
        $$
        $$
        =\left[\begin{array}{ll}
        12.60\angle85.45^\circ & 12.72\times10^3\angle80.95^\circ \\
        12.56\times10^{-3}\angle90^\circ & 12.60\angle85.45^\circ
        \end{array}\right]
        $$
    \end{description}

\newpage

実験によって算出した値と,理論値の比較を行ったものを次の表1,表2,表3,表4,表5,表6に示す.

\begin{table}[!ht]
    \centering
    \caption{被測定回路1におけるF行列の大きさの理論値と測定値の比較}
    \begin{tabular}{c|ccc}
    \hline
        & 理論値 & 測定値 & 相対誤差率[\%] \\ \hline
        $|A|$ & 6.36 & 7.04 & 10.69 \\ 
        $|B|$ & 1000.00 & 961.54 & -3.85 \\ 
        $|C|$ & 0.01 & 0.01 & 10.83 \\ 
        $|D|$ & 1.00 & 1.03 & 3.00 \\ \hline
    \end{tabular}
\end{table}

\begin{table}[!ht]
    \centering
    \caption{被測定回路1におけるF行列の偏角の理論値と測定値の比較}
    \begin{tabular}{c|ccc}
    \hline
        & 理論値 & 測定値 & 相対誤差率[\%] \\ \hline
        $\theta_A$ & 80.95 & 85.50 & 5.62 \\ 
        $\theta_B$ & 0.00 & 0.00 & 0.00 \\ 
        $\theta_C$ & 90.00 & 82.80 & -8.00 \\
        $\theta_D$ & 0.00 & 0.00 & 0.00 \\ \hline
    \end{tabular}
\end{table}

\begin{table}[!ht]
    \centering
    \caption{被測定回路2におけるF行列の大きさの理論値と測定値の比較}
    \begin{tabular}{c|ccc}
    \hline
        & 理論値 & 測定値 & 相対誤差率[\%] \\ \hline
        $|A|$ & 1.00 & 1.00 & 0.00 \\ 
        $|B|$ & 1000.00 & 794.12 & -20.59 \\ 
        $|C|$ & 0.01 & 0.01 & 0.80 \\ 
        $|D|$ & 6.36 & 5.44 & -14.47 \\ \hline
    \end{tabular}
\end{table}

\begin{table}[!ht]
    \centering
    \caption{被測定回路2におけるF行列の偏角の理論値と測定値の比較}
    \begin{tabular}{c|ccc}
    \hline
        & 理論値 & 測定値 & 相対誤差率[\%] \\ \hline
        $\theta_A$ & 0.00 & 0.00 & 0.00 \\ 
        $\theta_B$ & 0.00 & 0.00 & 0.00 \\ 
        $\theta_C$ & 90.00 & 90.00 & 0.00 \\
        $\theta_D$ & 80.95 & 86.40 & 6.73 \\ \hline
    \end{tabular}
\end{table}

\newpage

\begin{table}[!ht]
    \centering
    \caption{被測定回路3におけるF行列の大きさの理論値と測定値の比較}
    \begin{tabular}{c|ccc}
    \hline
        & 理論値 & 測定値 & 相対誤差率[\%] \\ \hline
        $|A|$ & 12.60 & 11.61 & -7.86 \\ 
        $|B|$ & 12720.00 & 5290.00 & -58.41 \\ 
        $|C|$ & 0.01 & 0.01 & 2.71 \\ 
        $|D|$ & 12.60 & 5.71 & -54.68 \\ \hline
    \end{tabular}
\end{table}

\begin{table}[!ht]
    \centering
    \caption{被測定回路3におけるF行列の偏角の理論値と測定値の比較}
    \begin{tabular}{c|ccc}
    \hline
        & 理論値 & 測定値 & 相対誤差率[\%] \\ \hline
        $\theta_A$ & 85.45 & 86.40 & 1.11 \\ 
        $\theta_B$ & 80.95 & 90.00 & 11.18 \\
        $\theta_C$ & 90.00 & 82.80 & -8.00 \\
        $\theta_D$ & 85.45 & 90.00 & 5.32 \\ \hline
    \end{tabular}
\end{table}

被測定回路1におけるF行列の大きさと偏角は,理論値と測定値の相対誤差率が
$-8.00\,\% \sim 10.69\,\%$に収まる結果となった.
本実験で使用しているコンデンサは誤差が
$\pm10\,\%$含んでいることを考慮するとうまく測定できたと判断できる.

被測定回路2におけるF行列の大きさは,理論値と測定値の相対誤差率が
$-20.59\,\% \sim 0.80\,\%$に収まる結果となった.
これは,コンデンサの誤差を踏まえたとしても,相対誤差率が大きい結果となった.
F行列の大きさ$|B||D|$を調べる際に,$V_1,V_{10},V_{20}$を計測するが,
その際にオシロスコープの画面を直接読み取ったため誤差が発生したと考えられる.
また,F行列の偏角は,理論値と測定値の相対誤差率が
$0.00\,\% \sim 6.73\,\%$に収まる結果となった.
これは,コンデンサの誤差が$\pm10\,\%$含んでいることを考慮するとうまく測定できた
と判断できる.

被測定回路3におけるF行列の大きさは,理論値と測定値の相対誤差率が
$-58.68\,\% \sim 2.71\,\%$に収まる結果となった.
これは,コンデンサの誤差を踏まえたとしても,相対誤差率が大きい結果となった.
F行列の大きさ$|B||D|$を調べる際に,$V_1,V_{10},V_{20}$を計測するが,
その際にオシロスコープの画面を直接読み取ったため誤差が発生したと考えられる.
また,F行列の偏角は,理論値と測定値の相対誤差率が
$-8.00\,\% \sim 11.18\,\%$に収まる結果となった.
これは,コンデンサの誤差が$\pm10\,\%$含んでいることを考慮するとうまく測定できた
と判断できる.

\newpage

\subsection{入出力端子を入れ替えた場合の関係}
被測定回路$1,2$の4端子定数の測定値より,入力端子と出力端子を入れ換えた場合の関係式
$$
A_2=D_1, B_2=B_1, C_2=C_1, D_2=A_1
$$
が成り立っているかどうかを検討する.

実験によって算出した値の比較を行ったものを次の表7,表8,表9,表10に示す.

\begin{table}[!ht]
    \centering
    \caption{$A_2=D_1$の比較}
    \begin{tabular}{cc|c}
    \hline
        $A_2$ & $D_1$ & 相対誤差率[\%] \\ \hline
        1.00 & 1.03 & 3.00 \\ \hline
    \end{tabular}
\end{table}

\begin{table}[!ht]
    \centering
    \caption{$B_2=B_1$の比較}
    \begin{tabular}{cc|c}
    \hline
        $B_2$ & $B_1$ & 相対誤差率[\%] \\ \hline
        794.12 & 961.54 & 21.08 \\ \hline
    \end{tabular}
\end{table}

\begin{table}[!ht]
    \centering
    \caption{$C_2=C_1$の比較}
    \begin{tabular}{cc|c}
    \hline
        $C_2$ & $C_1$ & 相対誤差率[\%] \\ \hline
        0.01 & 0.01 & 9.95 \\ \hline
    \end{tabular}
\end{table}

\begin{table}[!ht]
    \centering
    \caption{$D_2=A_1$の比較}
    \begin{tabular}{cc|c}
    \hline
        $D_2$ & $A_1$ & 相対誤差率[\%] \\ \hline
        5.44 & 7.04 & 29.41 \\ \hline
    \end{tabular}
\end{table}

$A_2=D_1$及び$C_2=C_1$の関係式は,相対誤差率が
$3.00\,\%$ , $9.95\,\%$となった.
これは,コンデンサの誤差が$\pm10\,\%$含んでいることを考慮すると関係式は
成り立っていると判断できる.

$B_2=B_1$及び$D_2=A_1$の関係式は,相対誤差率が
$21.08\,\%$ , $29.41\,\%$となった.
これは,コンデンサの誤差を踏まえたとしても,相対誤差率が大きい結果となった.
関係式は成り立っていないと判断できる.

\newpage

\subsection{自然回路としての関係式}
被測定回路$1,2,3$の4端子定数$(A, B, C, D)$の測定値より
$$
A D-B C=1
$$
が成り立っているかどうか検討する. 

それぞれの被測定回路における自然回路としての関係を次の表11,表12,表13に示す.

\begin{table}[!ht]
    \centering
    \caption{被測定回路1における自然回路としての関係}
    \begin{tabular}{cc|c}
    \hline
        理論値 & 測定値 & 相対誤差率[\%] \\ \hline
        1.00 & 0.65 & -35.00 \\ \hline
    \end{tabular}
\end{table}

\begin{table}[!ht]
    \centering
    \caption{被測定回路2における自然回路としての関係}
    \begin{tabular}{cc|c}
    \hline
        理論値 & 測定値 & 相対誤差率[\%] \\ \hline
        1.00 & 0.41 & -59.00 \\ \hline
    \end{tabular}
\end{table}

\begin{table}[!ht]
    \centering
    \caption{被測定回路3における自然回路としての関係}
    \begin{tabular}{cc|c}
    \hline
        理論値 & 測定値 & 相対誤差率[\%] \\ \hline
        1.00 & -1.95 & -295.00 \\ \hline
    \end{tabular}
\end{table}

被測定回路$1,2,3$で$A D-B C=1$の関係は,相対誤差率が
$35.00\,\%$,$59.00\,\%$,$-295.00\,\%$となった.
これは,コンデンサの誤差を踏まえたとしても,相対誤差率が大きい結果となった.
自然回路としての関係式は成り立っていないと判断できる.

\newpage

\subsection{縦続接続}
被測定回路3は被測定回路$1,2$を縦続接続したものである.縦続接続の場合の関係式
$$
\left[\begin{array}{ll}
A_3 & B_3 \\
C_3 & D_3
\end{array}\right]=\left[\begin{array}{ll}
A_1 & B_1 \\
C_1 & D_1
\end{array}\right]\left[\begin{array}{cc}
A_2 & B_2 \\
C_2 & D_2
\end{array}\right]
$$
が成り立っているかどうか検討する.

\begin{table}[!ht]
    \centering
    \caption{縦続接続の関係}
    \begin{tabular}{c|ccc}
    \hline
        & 理論値 & 測定値 & 相対誤差率[\%] \\ \hline
        $A_3$ & 12.60 & 13.22 & 4.92 \\ 
        $B_3$ & 12720.00 & 10890.00 & -14.39 \\ 
        $C_3$ & 0.01 & 0.01 & 7.32 \\ 
        $D_3$ & 12.60 & 11.13 & -11.67 \\ \hline
    \end{tabular}
\end{table}

$A_3$及び$C_3$の縦続接続の関係式は,相対誤差率が
$4.92\,\%$ , $7.32\,\%$となった.
これは,コンデンサの誤差が$\pm10\,\%$含んでいることを考慮すると関係式は
成り立っていると判断できる.

$B_3$及び$D_3$の関係式は,相対誤差率が
$-14,39\,\%$ , $-11.67\,\%$となった.
これは,コンデンサの誤差を踏まえたとしても,相対誤差率が大きい結果となった.
関係式は成り立っていないと判断できる.