% !TEX root = main.tex

%%%%%%%%%%%%%%%%%%%%%%%%%%%%%%%%%%%%%%%%%%%%%%%%%%%%%%%%%%%%%%%%%%%%%%%%%%%%%%%%%%%%%%%%%%%%%%%%
\section{理論}
%%%%%%%%%%%%%%%%%%%%%%%%%%%%%%%%%%%%%%%%%%%%%%%%%%%%%%%%%%%%%%%%%%%%%%%%%%%%%%%%%%%%%%%%%%%%%%%%

図 1 に示す回路で,最初,スイッチ S はの側に倒してあり,
十分長い時間が経過しているもの とする.
キャパシタが直流電圧 $-V$ で充電されている状態で,
時刻 $t=0$ において、スイッチ S を B 
の側へ切り替えて直流電圧 $V$ を印加する.印加される電圧の時間変化は,
図 2 の階段状の関数で表 される.

この場合の回路方程式は,回路に流れる電流を $i(t)$ として, $t>0$ で
\begin{align}
    L \frac{d i(t)}{d t}+R i(t)+\frac{1}{C} \int i(t) d t=V
\end{align}
となり,これよりキャパシタの電荷 $q(t)$ に関する微分方程式
\begin{align}
    L \frac{d^2 q(t)}{d t^2}+R \frac{d q(t)}{d t}+\frac{1}{C} q(t)=V
\end{align}
が得られる.

(2) 式の 2 階線形微分方程式は,特性方程式 $L p^2+R p+1 / C=0$ 
の解の判別条件に応じて次の解 をもつ.
ただし, $C_1, C_2$ は初期条件で決まる定数である.

\begin{enumerate}
    \item $R^2>4L/C$のとき
        \begin{align}
            q(t)=C_1 \exp \{(-\alpha+\beta) t\}+C_2 \exp \{(-\alpha-\beta) t\}+C V
        \end{align}
        \begin{align*}
            ただし \quad \alpha=\frac{R}{2 L}, \quad \beta=\sqrt{\left(\frac{R}{2 L}\right)^2-\frac{1}{L C}}
        \end{align*}

    \item $R^2=4 L / C$ のとき
        \begin{align}
            q(t)=\left(C_1+C_2 t\right) \exp (-\alpha t)+C V
        \end{align}

    \item $R^2<4 L / C$ のとき
        \begin{align}
            q(t)=C_1 \exp (-\alpha t) \sin \left(\beta t+C_2\right)+C V\quad(減㐮振動解)
        \end{align}
        \begin{align*}
            ただし \quad \alpha=\frac{R}{2 L}, \quad \beta=\sqrt{\frac{1}{L C}-\left(\frac{R}{2 L}\right)^2}
        \end{align*}
\end{enumerate}

(5) 式は $q(t)=C_1^{\prime} \exp (-\alpha t) \sin \beta t+C_2^{\prime} \exp (-\alpha t) \cos \beta t+C V$
 と表すこともできる.