% !TEX root = main.tex

%%%%%%%%%%%%%%%%%%%%%%%%%%%%%%%%%%%%%%%%%%%%%%%%%%%%%%%%%%%%%%%%%%%%%%%%%%%%%%%%%%%%%%%%%%%%%%%%
\section{考察}
%%%%%%%%%%%%%%%%%%%%%%%%%%%%%%%%%%%%%%%%%%%%%%%%%%%%%%%%%%%%%%%%%%%%%%%%%%%%%%%%%%%%%%%%%%%%%%%%

$v_c=0.623E$となる時刻に関する理論値と測定値の比較を表1,表2に示す.

\begin{table}[H]
    \centering
    \caption{測定回路1における理論値と測定値の比較}
    \begin{tabular}{cc|c}
    \hline
        理論値\,[$t$] & 測定値\,[$t$] & 相対誤差率\,[\%] \\ \hline
        $7.60\times10^{-6}$ & $7.50\times10^{-6}$ & -1.32 \\ \hline
    \end{tabular}
\end{table}

\begin{table}[H]
    \centering
    \caption{測定回路2における理論値と測定値の比較}
    \begin{tabular}{cc|c}
    \hline
        理論値\,[$t$] & 測定値\,[$t$] & 相対誤差率\,[\%] \\ \hline
        $4.00\times10^{-7}$ & $4.50\times10^{-7}$ & 12.50 \\ \hline
    \end{tabular}
\end{table}

測定回路1については,理論値と測定値の相対誤差率が-1.32\,[\%]
となった.本実験で使用しているコンデンサは誤差が\pm10[\%]含んでいることを
考慮するとうまく測定できたと判断できる.

測定回路2については,理論値と測定値の相対誤差率が
12.50\,[\%]となった.
これは,コンデンサの誤差を踏まえたとしても,相対誤差率が大きい結果となった.
$v_c=0.623E$となる時刻を調べる際,波形の傾きが大きく,
時間軸のオーダーが大きすぎたため,オシロスコープの画面を読み取るときに
誤差が発生したと考えられる.
