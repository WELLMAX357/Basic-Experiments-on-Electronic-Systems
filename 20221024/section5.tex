% !TEX root = main.tex

%%%%%%%%%%%%%%%%%%%%%%%%%%%%%%%%%%%%%%%%%%%%%%%%%%%%%%%%%%%%%%%%%%%%%%%%%%%%%%%%%%%%%%%%%%%%%%%%
\section{考察}
%%%%%%%%%%%%%%%%%%%%%%%%%%%%%%%%%%%%%%%%%%%%%%%%%%%%%%%%%%%%%%%%%%%%%%%%%%%%%%%%%%%%%%%%%%%%%%%%

\subsection{理論値と測定値の比較}
実験によって算出した値と,理論値の比較を行ったものを表2,表3,表4に示す.

\begin{table}[H]
    \centering
    \caption{インピーダンスの理論値と測定値の比較}
    \begin{tabular}{c|cc|c}
    \hline
        測定周波数$f$\,[Hz] & Zの理論値\,[\Omega] & Zの測定値\,[\Omega] & 相対誤差率\,[\%] \\ \hline
        100 & 0.846 & 0.933 & 10.36 \\ 
        1000 & 6.132 & 5.667 & -7.59 \\ 
        1600 & 9.726 & 9.333 & -4.04 \\ 
        2600 & 15.282 & 14.667 & -4.03 \\ 
        3600 & 19.367 & 18.667 & -3.62 \\ 
        4100 & 20.766 & 20.000 & -3.69 \\ 
        5100 & 21.717 & 20.667 & -4.84 \\ 
        6100 & 21.066 & 20.000 & -5.06 \\ 
        8900 & 16.830 & 16.667 & -0.97 \\ 
        9900 & 15.415 & 15.333 & -0.53 \\ 
        10900 & 14.168 & 13.333 & -5.89 \\
        50000 & 3.182 & 3.333 & 4.76 \\ 
        100000 & 1.591 & 1.667 & 4.73 \\ \hline
    \end{tabular}
\end{table}

\begin{table}[H]
    \centering
    \caption{インピーダンスの偏角の理論値と測定値の比較}
    \begin{tabular}{c|cc|c}
    \hline
        測定周波数$f$\,[Hz] & $\theta$の理論値\,[$^\circ$] & $\theta$の測定値\,[$^\circ$] & 相対誤差率\,[\%] \\ \hline
        100 & 44.700 & 136.800 & 206.04 \\
        1000 & 68.218 & 68.400 & 0.27 \\ 
        1600 & 60.025 & 60.480 & 0.76 \\ 
        2600 & 43.201 & 42.120 & -2.50 \\ 
        3600 & 24.815 & 22.680 & -8.60 \\ 
        4100 & 15.706 & 16.236 & 3.37 \\ 
        5100 & -1.056 & 0.000 & -100.00 \\ 
        6100 & -14.946 & -15.372 & 2.85 \\ 
        8900 & -39.806 & -38.448 & -3.41 \\ 
        9900 & -45.324 & -42.768 & -5.64 \\ 
        10900 & -49.772 & -51.012 & 2.49 \\ 
        50000 & -81.683 & -85.500 & 4.67 \\ 
        100000 & -85.852 & -97.200 & 13.22 \\ \hline
    \end{tabular}
\end{table}

\begin{table}[H]
    \centering
    \caption{電流相対比の理論値と測定値の比較}
    \begin{tabular}{c|cc|c}
    \hline
        測定周波数$f$\,[Hz] & $I_{ratio}$の理論値 & $I_{ratio}$の測定値 & 相対誤差率\,[\%] \\ \hline
        100 & 0.038 & 0.042 & 10.48 \\ 
        1000 & 0.279 & 0.258 & -7.59 \\ 
        1600 & 0.442 & 0.424 & -4.04 \\ 
        2600 & 0.695 & 0.667 & -4.03 \\ 
        3600 & 0.885 & 0.848 & -4.11 \\ 
        4100 & 0.944 & 0.909 & -3.69 \\ 
        5100 & 0.987 & 0.939 & -4.84 \\ 
        6100 & 0.958 & 0.909 & -5.06 \\ 
        8900 & 0.765 & 0.758 & -0.97 \\ 
        9900 & 0.701 & 0.697 & -0.53 \\ 
        10900 & 0.644 & 0.606 & -5.89 \\ 
        50000 & 0.145 & 0.152 & 4.76 \\ 
        100000 & 0.077 & 0.076 & -1.36 \\ \hline
    \end{tabular}
\end{table}

インピーダンスの大きさについては,理論値と測定値の相対誤差率が-7.59\,[\%]\sim10.36\,[\%]に収まる
結果となった.本実験で使用しているコンデンサは誤差が\pm10[\%]含んでいることを
考慮するとうまく測定できたと判断できる.

インピーダンスの偏角については,理論値と測定値の相対誤差率が
-100.00\,[\%]\sim206.04\,[\%]に収まる結果となった.
これは,コンデンサの誤差を踏まえたとしても,相対誤差率が大きい結果となった.
インピーダンスの偏角を調べる際に,$\Delta T$を計測するが,
その際にオシロスコープの画面を直接読み取ったため誤差が発生したと考えられる.

電流相対比について,理論値と測定値の相対誤差率が-7.59\,[\%]\sim10.48\,[\%]に収まる結果となった.

これは,コンデンサの誤差が\pm10\,[\%]
含んでいることを考慮すると適切に測定できていると判断できる.

\subsection{Sの実験値及びQの理論値の比較}
式(6)に実験から求めた数値を代入すると.