% !TEX root = main.tex

%%%%%%%%%%%%%%%%%%%%%%%%%%%%%%%%%%%%%%%%%%%%%%%%%%%%%%%%%%%%%%%%%%%%%%%%%%%%%%%%%%%%%%%%%%%%%%%%
\section{原理}
%%%%%%%%%%%%%%%%%%%%%%%%%%%%%%%%%%%%%%%%%%%%%%%%%%%%%%%%%%%%%%%%%%%%%%%%%%%%%%%%%%%%%%%%%%%%%%%%

\subsection{電場の測定}
一対の電極が作る電場を$\vec{E}(x, y, z)$とすると,$\vec{E}$の方向は等電位面
との直交条件を使って求められる.さらに,2つの等電位面間の電位差$\Delta V$と距離
$\Delta d$より,$\vec{E}$の大きさは以下の式で求められる.
$$
|\vec{E}|=-\frac{\Delta V}{\Delta d}
$$
ただし,上記の方法は電極により一様な空間に作られた電場に沿って電流が流れていること
を前提としており,障害物の近傍では電流路が著しく制限されることから,一様な空間に
作られる電場とは異なるものになる.そのため,本実験では障害物の影響も測定する.

\subsection{ガウスの法則}
電極板周りの空間の誘電率を$\varepsilon$とし,電極版を取り囲む任意の閉曲面を$S$,
閉曲面内の全電荷量を$Q$,面電荷密度をそれぞれ$\sigma$,$-\sigma$とすると、ガウスの法則より以下の式が成り立つ.
$$
\int_S(\varepsilon \vec{E}) \cdot d \vec{S}=Q=\pm \sigma S
$$

\subsection{仮想的な電極版}
電極板の厚み$t$が$t\cong 0$と近似できるほど非常に薄く,さらには電極板の面積$S$
が$S \rightarrow \infty$と近似できるほど大きいと仮定する.この時,導体内部の電位
$V$はいたるところで一定で,$\vec{E}$は電極板を垂直につながるので,
$$
\vec{E} \| d \vec{S} \quad \text { かつ }|\vec{E}|=k(\mathrm{k} \text { は定数 })
$$
が成り立つ.(2)、(3)とより,面電荷密度$\sigma$,$-\sigma$に対応する電極板が作る
電場をそれぞれ$\vec{E}_{+}$,$\vec{E}_{-}$とすると,
$$
\left|\vec{E}_{+}\right|=\left|\vec{E}_{-}\right|=\frac{\sigma}{2 \varepsilon}
$$
と求められる.ただし,$\vec{E}_{+}$は電極板から涌き出す方向,$\vec{E}_{-}$は
吸い込み方向となる.さらに重ね合わせの原理より,電極板の電場$\vec{E}$ の大きさは,
$$
|\vec{E}|=\left|\vec{E}_{+}\right|+\left|\vec{E}_{-}\right|=\frac{\sigma}{\varepsilon}
$$
と求められ,その方向は電極板と垂直かつ,プラスの電荷が帯電した電極板からマイナスの
電荷が帯電した電極板へ向かう向きとなる.また,静電場条件より
$$
\nabla \times \vec{E}=0 \Leftrightarrow \vec{E}=-\nabla V \Leftrightarrow V=E_x
$$
となるので,等電位面は電極板と平行になる.
ただし,現実は仮想的な電極板とは異なり,電極板は有限長であり.多くの場
合厚みを無視することはできない.そのため,本実験では実際に何\%ずれが生じているのか
測定する.