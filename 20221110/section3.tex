% !TEX root = main.tex

%%%%%%%%%%%%%%%%%%%%%%%%%%%%%%%%%%%%%%%%%%%%%%%%%%%%%%%%%%%%%%%%%%%%%%%%%%%%%%%%%%%%%%%%%%%%%%%%
\section{実験}
%%%%%%%%%%%%%%%%%%%%%%%%%%%%%%%%%%%%%%%%%%%%%%%%%%%%%%%%%%%%%%%%%%%%%%%%%%%%%%%%%%%%%%%%%%%%%%%%

\subsection{実験器具}
木製台,プローブ支持台,ガラス製水槽,平行平板電極,静電プローブ,METRONIX MTR18-1 直流定電圧定電流電源,TEKTRONIX TBS1022 オシロスコープ

\subsection{実験方法}
\subsubsection{設置}
\begin{itemize}
    \item $xy$平面において平面を格子状に分割し,その各格子点での電位$V(x,y)$を測定する.
    そして,各格子点の各座標とその点での電位を表にまとめる.
    \item 水槽内に平行電極板を設置し,電極間が$30\,\si{cm}$になるように調整する.
    その後,横から見て電極が平行になっていること,原点に対して電極板が$x$方向,$y$方向へ
    ずれていないこと,互いの電極が回転していないか確認し,固定用ネジで平行平板を固定する.
    ただし,水槽の中心を原点,$x$方向を水槽の長辺方向,$y$方向を水槽の短辺方向,$z$方向
    を水槽の深さ方向とする.
    \item 静電プローブ固定用治具に静電プローブを固定し,静電プローブから出ている配線をオ
    シロスコープのチャンネル1へ接続し,オシロスコープの設定を行う.
    \item 直流電源を準備し,直流電源の+極と-極をそれぞれケーブルで電極板へ接続する.
    さらに,直流電源のー極とオシロスコープのGNDを接続し,電極間に$10\,V$の電圧を印加する.    
\end{itemize}

\subsubsection{電位分布の測定}
\begin{itemize}
    \item $xy$平面において平面を格子状に分割し,その各格子点での電位$V(x,y)$を測定する.
    そして各格子点の各座標とその点での電位を表にまとめる.
\end{itemize}

\subsubsection{エッジ効果}
\begin{itemize}
    \item $xy$平面において電極板近傍の電位を詳細に測定し,その座標と電位を表にまとめる.
    ただし,実験課題2では実験課題1・3と異なり,電極板の端の中心を原点とした,$x^\prime$座標、$y^\prime$座標をとる.
    さらに,$x^\prime$方向を水槽の長辺方向,$y^\prime$方向を水槽の短辺方向,$z^\prime$方向を水槽の深さ方向とする.
\end{itemize}

\subsubsection{障害物の影響}
\begin{itemize}
    \item 水槽の中の適当な位置に金属を置き,$xy$平面の電位分布$V(x,y)$を表にまとめる.
    その際,金属を置いた座標を記録しておく.
\end{itemize}