% !TEX root = main.tex

%%%%%%%%%%%%%%%%%%%%%%%%%%%%%%%%%%%%%%%%%%%%%%%%%%%%%%%%%%%%%%%%%%%%%%%%%%%%%%%%%%%%%%%%%%%%%%%%

%%%%%%%%%%%%%%%%%%%%%%%%%%%%%%%%%%%%%%%%%%%%%%%%%%%%%%%%%%%%%%%%%%%%%%%%%%%%%%%%%%%%%%%%%%%%%%%%


\begin{landscape}
    \section{結果}
    \subsection{実験課題1}
    実験課題1の実験結果を表1に示す.ただし,$x$座標,$y$座標に対応する数値の単位は[V]とする.
    \begin{table}[H]
        \centering
        \caption{電位分布}
        \begin{tabular}{c|ccccccccccccccc}
        \hline
            \diagbox{$y$\,[\si{mm}]}{$x$\,[\si{mm}]} & -210 & -180 & -150 & -120 & -90 & -60 & -30 & 0 & 30 & 60 & 90 & 120 & 150 & 180 & 210 \\ \hline
            -120 & 1.88 & 1.96 & 2.16 & 2.44 & 2.68 & 3.14 & 3.48 & 3.90 & 4.76 & 5.12 & 5.50 & 5.76 & 6.08 & 5.92 & 6.42 \\ 
            -60 & 1.64 & 1.50 & 1.36 & 1.88 & 2.70 & 3.10 & 3.44 & 3.94 & 4.74 & 5.32 & 5.66 & 6.34 & 7.04 & 6.26 & 6.90 \\ 
            0 & 1.54 & 1.30 & 1.00 & 1.62 & 2.64 & 3.12 & 3.42 & 3.98 & 4.64 & 5.40 & 5.70 & 6.32 & 7.30 & 6.52 & 7.06 \\ 
            60 & 1.64 & 1.54 & 1.34 & 1.90 & 2.74 & 3.14 & 3.46 & 3.94 & 4.66 & 5.50 & 5.40 & 6.16 & 7.34 & 6.54 & 7.04 \\ 
            120 & 1.88 & 1.96 & 2.20 & 2.38 & 2.84 & 3.12 & 3.58 & 4.08 & 4.70 & 5.34 & 5.34 & 5.78 & 6.36 & 6.10 & 6.78 \\ \hline
        \end{tabular}
    \end{table}
\end{landscape}



\begin{landscape}
    \subsection{実験課題2}
    実験課題2の実験結果を表2に示す.ただし,$x$座標,$y$座標に対応する数値の単位は[V]とする.
    \begin{table}[!ht]
        \centering
        \caption{電極板近傍布巾の電位分布}
        \begin{tabular}{c|ccccccccc}
        \hline
            \diagbox{$y$\,[\si{mm}]}{$x$\,[\si{mm}]} & 110 & 120 & 130 & 140 & 150 & 160 & 170 & 180 & 190 \\ \hline
            50 & 6.82 & 6.98 & 6.98 & 7.58 & ~ & 8.02 & 7.70 & 7.68 & 7.56 \\ 
            60 & 6.78 & 6.88 & 7.12 & 7.36 & ~ & 8.06 & 7.64 & 7.56 & 7.44 \\ 
            70 & 6.62 & 6.84 & 6.98 & 6.84 & 6.52 & 7.84 & 7.50 & 7.48 & 7.40 \\ 
            80 & 6.30 & 6.82 & 7.02 & 6.56 & 6.32 & 7.42 & 7.34 & 7.40 & 7.32 \\ 
            90 & 6.60 & 6.70 & 6.86 & 6.50 & 6.14 & 7.12 & 7.16 & 7.30 & 7.18 \\ 
            100 & 6.56 & 6.62 & 6.72 & 6.38 & 6.12 & 6.80 & 7.14 & 6.94 & 7.20 \\ 
            110 & 6.50 & 6.54 & 6.68 & 6.18 & 6.16 & 6.64 & 7.04 & 7.08 & 7.02 \\ 
            120 & 6.46 & 6.50 & 6.60 & 6.20 & 6.06 & 6.76 & 7.00 & 7.00 & 7.06 \\ \hline
        \end{tabular}
    \end{table}
\end{landscape}



\begin{landscape}
    \subsection{実験課題3}
    実験課題3の実験結果を表3に示す.障害物の中心点は原点である.ただし,$x$座標,$y$座標に対応する数値の単位は[V]とする.
    \begin{table}[!ht]
        \centering
        \caption{障害物ありの電位分布}
        \begin{tabular}{c|ccccccccccccccc}
        \hline
            \diagbox{$y$\,[\si{mm}]}{$x$\,[\si{mm}]} & -210 & -180 & -150 & -120 & -90 & -60 & -30 & 0 & 30 & 60 & 90 & 120 & 150 & 180 & 210 \\ \hline
            -120 & 1.86 & 1.92 & 2.26 & 2.16 & 2.86 & 3.24 & 3.84 & 4.26 & 4.74 & 5.22 & 5.50 & 6.22 & 6.50 & 5.74 & 5.78 \\ 
            -60 & 1.76 & 1.48 & 1.32 & 1.98 & 2.64 & 3.16 & 3.68 & 4.08 & 4.76 & 5.32 & 5.56 & 6.58 & 7.68 & 6.68 & 5.88 \\ 
            0 & 1.60 & 1.22 & 0.98 & 1.70 & 2.68 & 3.12 & 3.50 & ~ & 5.10 & 5.52 & 5.84 & 6.74 & 7.86 & 6.80 & 6.12 \\ 
            60 & 1.66 & 1.56 & 1.28 & 1.90 & 2.76 & 3.10 & 3.62 & 4.30 & 4.92 & 5.42 & 5.70 & 6.56 & 7.58 & 6.60 & 6.10 \\ 
            120 & 1.84 & 1.94 & 2.22 & 2.34 & 2.92 & 3.24 & 3.72 & 4.34 & 4.88 & 5.26 & 5.72 & 5.92 & 6.66 & 6.02 & 5.98 \\ \hline
        \end{tabular}
    \end{table}
\end{landscape}
