% !TEX root = main.tex

%%%%%%%%%%%%%%%%%%%%%%%%%%%%%%%%%%%%%%%%%%%%%%%%%%%%%%%%%%%%%%%%%%%%%%%%%%%%%%%%%%%%%%%%%%%%%%%%
\section{考察}
%%%%%%%%%%%%%%%%%%%%%%%%%%%%%%%%%%%%%%%%%%%%%%%%%%%%%%%%%%%%%%%%%%%%%%%%%%%%%%%%%%%%%%%%%%%%%%%%

\begin{enumerate}
    \item 実験課題1で得られた結果をもとに,$xy$平面($z=0$)における等電位線の概要図
    を描け.図には電極板,印加電圧等の必要な情報も明記すること.また,等電位線は水槽壁
    面近傍および電極板の外側の領域についても,可能な限り詳細に記載すること.
    \begin{description}
        \item 概略図を図1に示す.
    \end{description}
    
    \item 実験課題1で得られた$xy$平面($z=0$)における測定値を用いて,各人で指定された電
    位に対する等電位線を最小2乗法による数値計算で求めよ.計算過程も明記すること.
    \begin{description}
        \item 近接する両サイド($x$方向)の測定値をそれぞれ$V_1$,$V_2$とし,
        その$x$座標をそれぞれ$x_1$,$x_2$とすると,$6.0\,\si{\volt}$となる
        座標$x$は,以下の式で導出される.
        $$
        x=\frac{|V_2-6|x_1+|V_1-6|x_2}{|V_2-V_1|}
        $$
        表1と上式より,導出結果を表4に示す
        また,$6.0\,\si{\volt}$に対する等電位線を
        $x = ay^2 + by + c$\,[mm]とおくと最小二乗法より,
        $$
        123
        $$
        以上より,$6.0\,\si{\volt}$に対する等電位線は
        $$
        x=0.00226y^2+0.0711y+69.0\,[\si{mm}]
        $$
        となる.
    \end{description}
    
    \item 設問(2)で得られた結果をグラフ用紙に描け.
    \begin{description}
        \item 結果を図2に示す.
    \end{description}

    \item 実験課題2で得た結果をもとに,電極板の端面近傍での電気力線を表す1次関数を
    1つ求めよ.
    \begin{description}
        \item ここで、7.5V に対する等電位線を
        $x=dy^2+ey+f$\,[\si{mm}]とおき,設問(3)と同様にデータ処理を行い,表5に示す.
        表5と最小二乗法より
        $$
        123
        $$
        
    \end{description}
    
\end{enumerate}