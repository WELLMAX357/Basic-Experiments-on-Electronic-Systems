% !TEX root = main.tex

%%%%%%%%%%%%%%%%%%%%%%%%%%%%%%%%%%%%%%%%%%%%%%%%%%%%%%%%%%%%%%%%%%%%%%%%%%%%%%%%%%%%%%%%%%%%%%%%
\section{考察}
%%%%%%%%%%%%%%%%%%%%%%%%%%%%%%%%%%%%%%%%%%%%%%%%%%%%%%%%%%%%%%%%%%%%%%%%%%%%%%%%%%%%%%%%%%%%%%%%

\begin{enumerate}
    \item 実験課題1で得られた結果をもとに,$xy$平面($z=0$)における等電位線の概要図
    を描け.図には電極板,印加電圧等の必要な情報も明記すること.また,等電位線は水槽壁
    面近傍および電極板の外側の領域についても,可能な限り詳細に記載すること.
    \begin{description}
        \item 概略図を図1に示す.
    \end{description}
    
    \item 実験課題1で得られた$xy$平面($z=0$)における測定値を用いて,各人で指定された電
    位に対する等電位線を最小2乗法による数値計算で求めよ.計算過程も明記すること.
    \begin{description}
        \item 近接する両サイド($x$方向)の測定値をそれぞれ$V_1$,$V_2$とし,
        その$x$座標をそれぞれ$x_1$,$x_2$とすると,$6.0\,\si{\volt}$となる
        座標$x$は,以下の式で導出される.
        $$
        x=\frac{|V_2-6|x_1+|V_1-6|x_2}{|V_2-V_1|}
        $$
        表1と上式より,導出結果を表4に示す
        また,$6.0\,\si{\volt}$に対する等電位線を
        $x = ay^2 + by + c$\,[mm]とおくと最小二乗法より,
        $$
        \left(\begin{array}{l}
        a \\
        b \\
        c
        \end{array}\right)=\left(\begin{array}{ccc}
        \sum y_i^4 & \sum y_i^3 & \sum y_i^2 \\
        \sum y_i^3 & \sum y_i^2 & \sum y_i^1 \\
        \sum y_i^2 & \sum y_i^1 & \sum y_i^0
        \end{array}\right)^{-1}\left(\begin{array}{c}
        \sum y_i^2 x_i \\
        \sum y_i^1 x_i \\
        \sum y_i^0 x_i
        \end{array}\right)=\left(\begin{array}{c}
        0.002485 \ldots \\
        -0.03671 \ldots \\
        99.13 \ldots
        \end{array}\right)
        $$
        以上より,$6.0\,\si{\volt}$に対する等電位線は
        $$
        x=0.00249y^2-0.0367y+99.1\,[\si{mm}]
        $$
        となる.
    \end{description}

\begin{landscape}
    \begin{table}[!ht]
        \centering
        \caption{6.0V 周りの電位分布とそのデータ処理}
        \begin{tabular}{c|c|cc|c|ccc|cc}
        \hline
             & \begin{tabular}{c} $y座標$ \\ $[\si{mm}]$ \end{tabular} & \begin{tabular}{c} $x_1=90\,\si{mm}$ \\ $上の電位V_1\,[\si{\volt}]$ \end{tabular} & \begin{tabular}{c} $x_2=150\,\si{mm}$ \\ $上の電位V_2\,[\si{\volt}]$ \end{tabular} & \begin{tabular}{c} $6.0\si{\volt}となる$ \\ $x座標\,[\si{mm}]$ \end{tabular} & \begin{tabular}{c} $y^2$ \\ $[\si{mm^2}]$ \end{tabular} & \begin{tabular}{c} $y^3$ \\ $[\si{mm^3}]$ \end{tabular} & \begin{tabular}{c} $y^4$ \\ $[\si{mm^4}]$ \end{tabular} & \begin{tabular}{c} $xy^2$ \\ $[\si{mm^3}]$ \end{tabular} & \begin{tabular}{c} $xy$ \\ $[\si{mm^2}]$ \end{tabular} \\ \hline
             & -120.00 & 5.50 & 6.08 & 141.72 & 14400.00 & -360.00 & 207360000.00 & 2040827.59 & -17006.90 \\ 
             & -60.00 & 5.66 & 7.04 & 104.78 & 3600.00 & -180.00 & 12960000.00 & 377217.39 & -6286.96 \\ 
             & 0.00 & 5.70 & 7.30 & 101.25 & 0.00 & 0.00 & 0.00 & 0.00 & 0.00 \\ 
             & 60.00 & 5.40 & 7.34 & 108.56 & 3600.00 & 180.00 & 12960000.00 & 390804.12 & 6513.40 \\ 
             & 120.00 & 5.34 & 6.36 & 128.82 & 14400.00 & 360.00 & 207360000.00 & 1855058.82 & 15458.82 \\ \hline
            合計 &  &  &  & 585.14 & 36000.00 & 0.00 & 440640000.00 & 4663907.92 & -1321.63 \\ \hline
        \end{tabular}
    \end{table}
\end{landscape}

    
        
    
    
    \item 設問(2)で得られた結果をグラフ用紙に描け.
    \begin{description}
        \item 結果を図2に示す.
    \end{description}

    \item 実験課題2で得た結果をもとに,電極板の端面近傍での電気力線を表す1次関数を
    1つ求めよ.
    \begin{description}
        \item ここで、7.5V に対する等電位線を
        $x=dy^2+ey+f$\,[\si{mm}]とおき,設問(3)と同様にデータ処理を行い,表5に示す.
        表5と最小二乗法より
        $$
        \left(\begin{array}{l}
        d \\
        e \\
        f
        \end{array}\right)=\left(\begin{array}{ccc}
        \sum y_i^4 & \sum y_i^3 & \sum y_i^2 \\
        \sum y_i^3 & \sum y_i^2 & \sum y_i^1 \\
        \sum y_i^2 & \sum y_i^1 & \sum y_i^0
        \end{array}\right)^{-1}\left(\begin{array}{c}
        \sum y_i^2 x_i \\
        \sum y_i^1 x_i \\
        \sum y_i^0 x_i
        \end{array}\right)=\left(\begin{array}{c}
        0.002485 \ldots \\
        -0.03671 \ldots \\
        99.13 \ldots
        \end{array}\right)
        $$
        以上より,$6.0\,\si{\volt}$に対する等電位線は
        $$
        x=0.00249y^2-0.0367y+99.1\,[\si{mm}]
        $$
        となる.
    \end{description}
    
    \item 実験で用いた水道水を純水に変えた場合,この実験はどのようになるか考察せよ.
    \begin{description}
        \item 水道水は不純物が溶け込んでいるため,電流を通す.一方で、純水は不純物
        がないため電気を通さない.実験で用いた水道水を純水に変えた場合,
        電極板間に電気が流れないため,電場は発生しないと考えられる.
    \end{description}

    \item もしも水槽内に水道水が入っていなかったら,$\vec{E}$の空間分布はどうなるか?
    理由と共にそのグラフの概略を示せ.    
    \begin{description}
        \item 水道水が入っていない場合,誘電体は空気となる.
        また,誘電体が変わった場合,誘電率は高くなるため$\vec{E}$の大きさは小さくなり,
        $\vec{E}$の向きは変わらない.$\vec{E}$の空間分布は実験課題1と変わらな
        いと考えられる.よって,概略図は図3のようになる.        
    \end{description}

    \item 実験課題3で得られた結果より等電位線を描け.
    \begin{description}
        \item 等電位線を図4に示す.
    \end{description}

\end{enumerate}