% !TEX root = main.tex

%%%%%%%%%%%%%%%%%%%%%%%%%%%%%%%%%%%%%%%%%%%%%%%%%%%%%%%%%%%%%%%%%%%%%%%%%%%%%%%%%%%%%%%%%%%%%%%%
\section{目的}
%%%%%%%%%%%%%%%%%%%%%%%%%%%%%%%%%%%%%%%%%%%%%%%%%%%%%%%%%%%%%%%%%%%%%%%%%%%%%%%%%%%%%%%%%%%%%%%%

等角写像法の原理を理解し,その技法から得られる等電位面と電気力線の関係を視覚的
に描く.また数値計算により,種々の形状をした電極電位が作り出す等電位面を求め,静電
場$\vec{E}$の様子を理解する.更には,実験で用いた平行平板電極のセットアップに即した等電位
面を数値計算ソフトを用いて可視化する.最後に,実験で得られた電位$V$の空間分布を解
析解比較し,電極エッジ効果を理解する.
