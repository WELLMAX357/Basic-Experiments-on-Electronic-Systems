% !TEX root = main.tex

%%%%%%%%%%%%%%%%%%%%%%%%%%%%%%%%%%%%%%%%%%%%%%%%%%%%%%%%%%%%%%%%%%%%%%%%%%%%%%%%%%%%%%%%%%%%%%%%
\section{原理}
%%%%%%%%%%%%%%%%%%%%%%%%%%%%%%%%%%%%%%%%%%%%%%%%%%%%%%%%%%%%%%%%%%%%%%%%%%%%%%%%%%%%%%%%%%%%%%%%

\subsection{コンデンサ放電}
大電流をソレノイド$L$に流すことで強い磁場$B$を発生させる.
このためには大きな電流$I$が必要となり,本実験ではコンデンサ放電を用いる.
コンデンサ放電では,コンデンサ$C$を高電圧で充電することにより大電荷$Q$
を貯めることができ,この$Q$を急速放電$\left(\frac{d}{d t} \gg 1\right)$
させることで,$\frac{d Q}{d t}=I$により大きな$I$を発生させる.

\subsection{RLC回路における過渡現象の原理}
キルヒホッフの第2法則よりRLC回路では以下の回路方程式が成り立つ.
$$
R i+L\frac{di}{dt}+\frac{1}{c} \int i d t=E
$$
ここで,$D=\left(\frac{R}{2 L}\right)^2-\frac{1}{L C}, \alpha=-\frac{R}{2 L} , \beta=\sqrt{\left(\frac{R}{2 L}\right)^2-\frac{1}{L C}}, \beta^{\prime}=\sqrt{-\left\{\left(\frac{R}{2 L}\right)^2-\frac{1}{L C}\right\}}$
とおくと,電流 $i$ は以下の式で描かれる。
$$
\begin{gathered}
\mathrm{D}>0 \text { のとき } i=\frac{E}{\beta L} e^{\alpha t} \sinh \beta t \\
\mathrm{D}<0 \text { のとき } i=\frac{E}{\beta^{\prime} L} e^{\alpha t} \sinh \beta^{\prime} t \\
\mathrm{D}=0 \text { のとき } i=\frac{E}{L} e^{\alpha t}
\end{gathered}
$$
このとき,電流$i$は$D>0$のとき過減衰,$D=0$のとき臨界減衰,$D<0$ のとき不足減衰
となる.


