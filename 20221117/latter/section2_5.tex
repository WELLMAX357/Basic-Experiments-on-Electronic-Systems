% !TEX root = main.tex

%%%%%%%%%%%%%%%%%%%%%%%%%%%%%%%%%%%%%%%%%%%%%%%%%%%%%%%%%%%%%%%%%%%%%%%%%%%%%%%%%%%%%%%%%%%%%%%%
\section{データ解析と考察}
%%%%%%%%%%%%%%%%%%%%%%%%%%%%%%%%%%%%%%%%%%%%%%%%%%%%%%%%%%%%%%%%%%%%%%%%%%%%%%%%%%%%%%%%%%%%%%%%
\begin{enumerate}
    \item 実験課題1で得られた$L$の実験値を理論値と比較せよ.
    なお,計算においては長岡係数を考慮すること.
    \begin{description}
        \item[] 
    \end{description}

    \item (1)で計算した$L$の理論値を用いて,臨界制動となるときの$R$の値を算出
    せよ.また,実験的に求めた$R$の値と比較し,$R$の値が異なるときは,その原因
    を考察せよ.
    \begin{description}
        \item[] 
    \end{description}
\end{enumerate}