% !TEX root = main.tex

%%%%%%%%%%%%%%%%%%%%%%%%%%%%%%%%%%%%%%%%%%%%%%%%%%%%%%%%%%%%%%%%%%%%%%%%%%%%%%%%%%%%%%%%%%%%%%%%
\section{データ解析と考察}
%%%%%%%%%%%%%%%%%%%%%%%%%%%%%%%%%%%%%%%%%%%%%%%%%%%%%%%%%%%%%%%%%%%%%%%%%%%%%%%%%%%%%%%%%%%%%%%%
\begin{enumerate}
    \item 実験課題1で得られた$L$の実験値を理論値と比較せよ.
    なお,計算においては長岡係数を考慮すること.
    \begin{description}
        \item[] ソレノイドコイルの自己インダクタンス$L$は,ソレノイドコイル
        の半径を$a$,長さを$b$として以下の式を用いて求めることができる.
        $$
        L=\pi \lambda a^2 \mu_s \mu_0 \frac{N^2}{b}
        $$
        また,コイルの直径と長さの比は以下の式を用いて求めることができる.
        $$
        k=\frac{2a}{b}
        $$
        ここで,ソレノイドコイルの測定では,$a = 0.0465\,[\si{m}], b = 0.250\,[\si{m}]$となった.
        このとき,コイルの直径と長さの比は
        $$
        k=\frac{2\times0.0465}{0.250}=0.372
        $$
        となる.テキストp54の表2:長岡係数表を参考にすると,求めた値$0.372$
        に近い$0.4$場合長岡係数$\lambda$は$0.850$と求められる.
        したがって,$\mu_0 = 1.257 \times 10^{-6}\,[\si{H/m}] , \mu_s = 1\,[\si{H/m}]$
        から,ソレノイドコイルの自己インダクタンス$L$は,
        $$
        L=\pi \times 0.850 \times (0.0465)^2 \times 1 \times 1.257 \times 10^{-6} \times \frac{92^2}{0.250} \fallingdotseq 246\,[\si{\mu H}]
        $$
        と求められる.ここで$L$の実験値と理論値を比較すると,その相対誤差率は,
        $$
        \frac{337.7-245.7}{245.7} \fallingdotseq 37.4\,[\%]
        $$
        となる.
    \end{description}

    \item (1)で計算した$L$の理論値を用いて,臨界制動となるときの$R$の値を算出
    せよ.また,実験的に求めた$R$の値と比較し,$R$の値が異なるときは,その原因
    を考察せよ.
    \begin{description}
        \item[] (1)で算出した$L$の理論値を用いると,抵抗値$R$の値は,
        $$
        R=\sqrt[]{\frac{4 \times 245.7 \times 10^{-6}}{12 \times 10^{-6}}} \fallingdotseq 9.05\,[\Omega]
        $$
        と求められる.ここで,$R$の実験値と理論値を比較すると,その相対誤差率は,
        $$
        \frac{10.60-9.049}{9.049} \fallingdotseq 17.1\,[\%]
        $$
        となる.

        この相対誤差率の原因として,周期$T$の読み取り誤差が考えられる.
        オシロスコープの横軸の一目盛りは$50\,\si{\mu s}$であり,読み取り誤差が$\pm 5\,\si{\mu s}$
        生じる可能性がある.この誤差が生じた際,抵抗値$R$は$\pm 0.13\,\si{\Omega}$
        の誤差が生じるため,相対誤差が発生した.
    \end{description}
\end{enumerate}