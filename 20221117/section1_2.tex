% !TEX root = main.tex

%%%%%%%%%%%%%%%%%%%%%%%%%%%%%%%%%%%%%%%%%%%%%%%%%%%%%%%%%%%%%%%%%%%%%%%%%%%%%%%%%%%%%%%%%%%%%%%%
\section{原理}
%%%%%%%%%%%%%%%%%%%%%%%%%%%%%%%%%%%%%%%%%%%%%%%%%%%%%%%%%%%%%%%%%%%%%%%%%%%%%%%%%%%%%%%%%%%%%%%%

\subsection{対角写像法}
電極に与えられた電位$V_0$が作る$V$の空間分布$V(x,y,z)$を求めるには,
ラプラス方程式
$$
\nabla^2 V=\frac{\partial^2 V}{\partial x^2}+\frac{\partial^2 V}{\partial y^2}+\frac{\partial^2 V}{\partial z^2}=0
$$

を2つの境界条件

\begin{itemize}
    \item 導体の電位は一定である.
    \item 誘電体の教会では,電束密度$\vec{D}$の法線方向の成分と$\vec{E}$の接線方向の成分が連続となる.
\end{itemize}

の下で解けば良い.この式は,$V$が2次元でかつ対称性が良い場合,
等角写像を応用することで綺麗に解くことができる.

等角写像とは,ある2つの複素変数$z= (x + iy)とw = (u + iv)$が正則な関数$f$
によって$w =f(z)$と結び付けられているとき,$z$の値を変化させて$z$平面上である
図形を描くと,それぞれの$z$の値に対応する$w$もまた$w$平面上である図形を描く.
このとき,$f(z)$が解析関数である限り,$z$平面上での3点が作る角度
$\angle z_1z_2z_3$と,$w$平面上の対応する3点が作る角度$\angle w_1w_2w_3$
が必ず等しくなる.このような写像を等角写像という.

また,平行平板電極が$x$軸と平行に$y = \pm d$の位置に置かれ,$y = d$の電極に
$V_0$,$y = -d$の電極に$-V_0$の電圧を印加した場合,$V(x,y)$の等電位線は以下の式で
表される軌跡となる.

$$
\begin{aligned}
&x=d\left\{t+\frac{1}{\pi} e^{\pi t} \cos \left(\frac{\pi V}{V_0}\right)\right\} \\
&y=d\left\{\frac{V}{V_0}+\frac{1}{\pi} e^{\pi t} \sin \left(\frac{\pi V}{V_0}\right)\right\}
\end{aligned}
$$

\subsection{数値計算}

数値計算では,数理モデル化によって表された微分方程式を離散化し,
それらによって得られた代数方程式をコンピュータを用いて解析する.

本実験では,Mesh及びEStatというソフトウェアを用いて計算機実験を行い,Meshでは,
有限要素法解析に必要な三角メッシュを生成し,EStatによりポテンシャルの計算を行う.

有限要素法とは,領域を多数の部分領域に分割し,その領域内において単純な関数の重ね
合わせによって未知量を近似する手法である.
