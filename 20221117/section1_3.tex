% !TEX root = main.tex

%%%%%%%%%%%%%%%%%%%%%%%%%%%%%%%%%%%%%%%%%%%%%%%%%%%%%%%%%%%%%%%%%%%%%%%%%%%%%%%%%%%%%%%%%%%%%%%%
\section{実験}
%%%%%%%%%%%%%%%%%%%%%%%%%%%%%%%%%%%%%%%%%%%%%%%%%%%%%%%%%%%%%%%%%%%%%%%%%%%%%%%%%%%%%%%%%%%%%%%%

\subsection{シミュレーション手順}
\begin{enumerate}
    \item Meshを用いて実験課題にあった領域を作成し,メッシュを作成する.
    \item Estateを用いて,実験課題にあった領域要素(電位,誘電率)を指定し,
    ポテンシャルの計算を行う.その後,解析結果を表示する.本実験では,比誘電率
    を以下のように定める.\\
    水:$80.4(20\si{\celsius})$\quad 空気:$1.00$\quad ガラス:$7.65$\\
    また,印加電圧は$10\,[\si{\volt}]$とする。
\end{enumerate}

\subsection{先週の実験を模擬する}
ソフトフェアを用いて,第一週目の実験セットアップとパラメータの時の等電位面を計
算し,$(x,y)$平面における等電位線を描く.

\subsection{境界の役割を学ぶ}
ソフトウェアを用いて,境界が無限遠とした場合の等電位面を計算する.なお,この計算
においては、電極の厚み$t$と間隔$d$は自由に設定してよい.

\subsection{真空中での電位分布}
ソフトウェアを用いて,媒質が真空の場合の等電位面を計算する.

\subsection{電極形状の影響}
ソフトウェアを用いて,平行平板電極の形状を曲面板とした場合の等電位面を計算する.
