% !TEX root = main.tex

%%%%%%%%%%%%%%%%%%%%%%%%%%%%%%%%%%%%%%%%%%%%%%%%%%%%%%%%%%%%%%%%%%%%%%%%%%%%%%%%%%%%%%%%%%%%%%%%
\section{宿題}
%%%%%%%%%%%%%%%%%%%%%%%%%%%%%%%%%%%%%%%%%%%%%%%%%%%%%%%%%%%%%%%%%%%%%%%%%%%%%%%%%%%%%%%%%%%%%%%%

\begin{enumerate}
    \item 第1週目で行った実験時のセットアップ値を(19)式
    に代入することで,貴方の実験グループの解析解を示せ.
    \begin{description}
        \item[] 第1週目に行った実験のパラメータは,
        $d = 150\,[\si{mm}],V_0 = 5\,[\si{\volt}]$であるので,解析解は次のようになる.
        $$
        x=150\left\{t+\frac{1}{\pi} e^{\pi t} \cos \left(\frac{\pi V}{5}\right)\right\}[\mathrm{mm}], y=150\left\{\frac{V}{5}+\frac{1}{\pi} e^{\pi t} \sin \left(\frac{\pi V}{5}\right)\right\}[\mathrm{mm}]
        $$
    \end{description}

    \item 実験課題3より分かるように,媒質が真空でも等電位面は存在しているが,
    真空中では実験で用いたプローブから信号出力を得ることはできない.
    この理由を説明せよ.
    \begin{description}
        \item[] 水の比誘電率が$80.4(20\si{\celsius})$であるのに対し,
        真空の比誘電率は$8.8541\times 10^{-12}$である.
        そのため媒質が水の場合と比べて,真空中では電場が非常に小さくなり,
        電場によって生じる電流が小さくなる.真空中では媒質中に流れる電流値が
        実験で用いたプローブの電流測定が可能な範囲よりも小さくなるため,
        信号出力を得ることはできないと考えられる.
    \end{description}
\end{enumerate}