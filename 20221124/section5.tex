% !TEX root = main.tex

%%%%%%%%%%%%%%%%%%%%%%%%%%%%%%%%%%%%%%%%%%%%%%%%%%%%%%%%%%%%%%%%%%%%%%%%%%%%%%%%%%%%%%%%%%%%%%%%
\section{データ解析と考察}
%%%%%%%%%%%%%%%%%%%%%%%%%%%%%%%%%%%%%%%%%%%%%%%%%%%%%%%%%%%%%%%%%%%%%%%%%%%%%%%%%%%%%%%%%%%%%%%%
\begin{enumerate}
    \item 実験課題1で得られた結果から,ソレノイド中心軸上の軸方向磁束密度分布
    $B_z(z)$を算出せよ.ただし,導出過程も示すこと.
    \begin{description}
        \item[] 各$z$座標における磁気プローブの出力$V_{co}$を積分し,
        $\int_{0}^{t} V_{co}(t)dt$を算出する.その後,ソレノイド中心軸上の軸方向磁束密度分
        布$B_z(z)$を以下の式で求め,算出結果を有効数字3桁で表3に示す。
        $$
        B=-\frac{1}{NS}\int_{0}^{t} V_{co}(t)dt
        $$
        ただし、各パラメータは次のようになる.
        $$
        N=33\,[回],S=\pi r^2=\pi \times 0.00275^2 \simeq 2.38\times 10^{-5}\,[\si{m^2}]
        $$

        %chart
    \end{description}

    \item ソレノイド中心軸上の軸方向磁束密度分布$B_z(z)$をビオ・サバールの法則
    を用いて計算せよ.ただし,ビオ・サバールの法則を明確に示し,そこからソレノイド
    中心軸上の磁束密度を求める導出過程も詳細に記述すること.
    \begin{description}
        \item[] ビオ・サバールの法則より以下の式が成り立つ.
        $$
        \vec{dB}=\frac{\mu_0 I}{4\pi}\frac{\vec{dl}\times\vec{r}}{|\vec{r}|^3}
        $$
        ここで,図1のような有限長ソレノイドについて考えると,単位長さあたりの
        電流は$nIdz$であるから,ビオ・サバールの法則より,
        $$
        dB=\frac{\mu_0}{4\pi}\frac{nIdzdl\sin\theta}{r^2}=-\frac{\mu_0}{4\pi}\frac{nI\sin\theta}{a}dld\theta
        $$
        となる。ここで$\int_{c}dl=2\pi ab$であり,
        $d\theta$を$\theta_1 \rightarrow \theta_2$で積分すると,
        以下の式が成り立つ.
        $$
        B=\frac{\mu_0 NI}{2}(\cos\theta_2 - \cos\theta_1)
        $$
        したがって,ソレノイド中心軸上の磁束密度は以下の式で求められ,
        その算出結果を有効数字3桁で表4に示す.
        $$
        B_z=\frac{\mu_0 NI}{2}\{\frac{z}{\sqrt[]{a^2+z^2}}+\frac{b-z}{\sqrt[]{a^2+(z-b)^2}}\}
        $$
        ただし,各パラメータは,
        $$
        N=33,\mu_0=1.257\times 10^{-6}\,[\si{H/m}],a=0.0465\,[\si{m}],b=0.250\,[\si{m}]
        $$
        となり,電流$I$は$R=10.6\,[\si{\Omega}]$に関して,$I=\frac{V_R}{R}$で求める.
        %chart
    \end{description}
    
    \item 設問(1)および(2)で得られた結果をグラフに重ね書きしなさい.そして,
    貴方のプローブ測定精度を有効数字2桁で示せ.
    \begin{description}
        \item[] 設問(1)および(2)で得られた結果を重ね書きしたグラフを図 に示す.
        プローブ測定精度を以下の式で求め,その算出結果を有効数字2桁で表5に示す.
        $$
        \frac{\delta B}{B}=\frac{実験値-理論値}{理論値}
        $$
    \end{description}

    \item 実験課題2において,抵抗$R$の両端の電圧波形$V_R$から閉ループ回路に
    流れるパルス電流の最大値を求めよ.導出過程も示すこと.
    \begin{description}
        \item[] 各鎖交数における電圧波形$V_R$におけるピーク電圧$V_{Rmax}$を
        読み取ることで,閉ループ回路に流れるパルス電流の最大値$IR$は以下の式で
        算出される.
        $$
        I_R=\frac{V_{Rmax}}{R}
        $$
        この算出結果を有効数字3桁で表6に示す.ただし,$R= 10.5\,[\Omega]$
        である.表6より,パルス電流の最大値の最確値は次のように求められる.
        $$
        I_{RS}=
        $$
    \end{description}
    
    \item 実験課題2において,ロゴスキーコイルの出力$V_e$から閉ループ回路に
    流れるパルス電流の最大値を求めよ.導出過程も示すこと.
    \begin{description}
        \item[] 閉ループ回路に流れるパルス電流は以下の式で算出される.
        $$
        i_e=-\frac{l}{\mu_0 NS}\int_{0}^{t}V_e(t)dt
        $$
        この算出結果を表7に示す.ただし,各パラメータは次のようになる.
        $$
        \mu_0=1.257\times 10^{-6}\,[\si{H/m}],N=219,S=\pi\times 0.11875^2\simeq 0.04430\,[\si{m^2}],l=2\pi\times 0.11875\simeq 0.7461\,[\si{m}]
        $$
        表7より,パルス電流の最大値の最確値は次のように求められる.
    \end{description}
    
\end{enumerate}