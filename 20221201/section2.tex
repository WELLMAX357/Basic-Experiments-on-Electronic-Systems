% !TEX root = main.tex

%%%%%%%%%%%%%%%%%%%%%%%%%%%%%%%%%%%%%%%%%%%%%%%%%%%%%%%%%%%%%%%%%%%%%%%%%%%%%%%%%%%%%%%%%%%%%%%%
\section{原理}
%%%%%%%%%%%%%%%%%%%%%%%%%%%%%%%%%%%%%%%%%%%%%%%%%%%%%%%%%%%%%%%%%%%%%%%%%%%%%%%%%%%%%%%%%%%%%%%%

変位電流密度$\vec{i_d}$とは電束$\vec{D}$の時間変化であり,
以下の式で与えられる.
$$
\vec{i_d}=\frac{\partial\vec{D}}{\partial t}
$$

また,平行平板への電圧限として交流を与え場合について考える.微小区間$\Delta x$
離れた二点での電位を測定すると電場は,$|\vec{E}|=\frac{\Delta V}{\Delta x}$
で計算でき,電束密度を$\vec{E}=\epsilon\vec{D}$と仮定できる.
したがって,変位電流密度$\vec{i_d}$は以下の式で与えられる.
$$
|\vec{i_d}|=|\frac{\partial\vec{D}}{\partial t}|=\epsilon|\frac{\partial}{\partial t}(\frac{\Delta V}{\Delta x})|=\frac{\epsilon}{\Delta x}|\frac{\partial}{\partial t}(\Delta V)|
$$
ここで,平行平板に印加する$V$の角周波数を$\omega$とすると,$V\propto \sin \omega t$と書けるので,
平行平板電極の面積を$S$.二点での電位をそれぞれ$V_1 = A\sin \omega t$,$V_2=B\sin \omega t$
とすると変位電流の大きさ$I_{dmax}$は以下の式で求められる.
$$
|I_{dmax}|=\frac{\epsilon}{\Delta x}|(A-B)\omega|
$$
また,ロゴスキーコイルにおいてロゴスキーコイルの両端に現れる誘導電圧を$V_e(t)=C\sin \omega t$
とすると,変位電流の大きさ$I_d$は以下の式で求められる.
$$
|I_{dmax}|=-|\frac{l}{\mu_0 NS}\int_{\frac{\pi}{2}}^{\pi}V_e(t)dt|=\frac{l}{\mu_0 NS}\cdot\frac{C}{\omega}
$$
