% !TEX root = main.tex

%%%%%%%%%%%%%%%%%%%%%%%%%%%%%%%%%%%%%%%%%%%%%%%%%%%%%%%%%%%%%%%%%%%%%%%%%%%%%%%%%%%%%%%%%%%%%%%%
\section{データ解析と考察}
%%%%%%%%%%%%%%%%%%%%%%%%%%%%%%%%%%%%%%%%%%%%%%%%%%%%%%%%%%%%%%%%%%%%%%%%%%%%%%%%%%%%%%%%%%%%%%%%

\begin{enumerate}
    \item 実験課題1について,二本のプローブを用いて得られた電位差$\Delta V$の時間変化データ
    $\Delta V(t)$と$\Delta x$より,それぞれの発振周波数における$I_d$の値を求め,表にして
    示せ.
    \begin{description}
        \item[] 算出した変位電流$I_d$を表1に示す.ただし,テキストp234より真空の誘電率は
        $8.854\times 10^{-12}\,[\si{F/m}]$であり,p38より水の比誘電率は
        $80.4\,(20\si{\celsius})$なので,$\epsilon=7.11862\times 10^{-10}\,[\si{F/m}]$
        として算出する.

    \end{description}

    \item 実験課題2において,それぞれの発振周波数におけるロゴスキーコイルの出力
    から$I_d$を算出し,表にして示せ.
    \begin{description}
        \item[] 算出した変位電流$I_d$を表2に示す.また,$\mu_0=0.000001257\,[\si{[H/m]}]$
        ,$N=211$,$l=2\pi\times 0.1095=0.688\,[\si{m}]$,
        $S=\pi\times 0.0100^2=0.000314\,[\si{m^2}]$である.
        
    \end{description}

    \item 二本のプローブを用いて得られた$I_d$とロゴスキーコイルから得られた$I_d$の値を
    比較検討せよ.
    \begin{description}
        \item[] 
    \end{description}
    
    \item 実験課題2について,ロゴスキーコイルから得られた$I_d$と,セメント抵抗の両端の電圧波形から
    得られる$I_F$の位相の相対関係を示せ.
    \begin{description}
        \item[] 
    \end{description}

    \item コンデンサーを含む回路では,$I_F$と$I_d$が閉ループを作るために,$I_F+I_d$はどのよう
    な面を取っても,$I_F+I_d=一定$となる.この予測が正しいかどうか得られた実験
    データに基づいて判定せよ.
    \begin{description}
        \item[] 
    \end{description}
    
\end{enumerate}