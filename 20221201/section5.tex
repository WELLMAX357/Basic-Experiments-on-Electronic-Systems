% !TEX root = main.tex

%%%%%%%%%%%%%%%%%%%%%%%%%%%%%%%%%%%%%%%%%%%%%%%%%%%%%%%%%%%%%%%%%%%%%%%%%%%%%%%%%%%%%%%%%%%%%%%%
\section{データ解析と考察}
%%%%%%%%%%%%%%%%%%%%%%%%%%%%%%%%%%%%%%%%%%%%%%%%%%%%%%%%%%%%%%%%%%%%%%%%%%%%%%%%%%%%%%%%%%%%%%%%

\begin{enumerate}
    \item 実験課題1について,二本のプローブを用いて得られた電位差$\Delta V$の時間変化データ
    $\Delta V(t)$と$\Delta x$より,それぞれの発振周波数における$I_d$の値を求め,表にして
    示せ.
    \begin{description}
        \item[] 算出した変位電流$I_d$を表1に示す.ただし,テキストp234より真空の誘電率は
        $8.854\times 10^{-12}\,[\si{F/m}]$であり,p38より水の比誘電率は
        $80.4\,(20\si{\celsius})$なので,$\epsilon=7.11862\times 10^{-10}\,[\si{F/m}]$
        として算出する.
        \begin{table}[H]
            \centering
            \caption{実験課題1における変位電流$I_d$}
            \begin{tabular}{c|c|c}
                \hline
                $f\,[\si{kHz}]$ & $\Delta V\,[\si{V}]$ & $I_{dmax}\,[\si{A}]$ \\ \hline
                1000 & 0.044 & 0.00028339 \\ 
                900 & 0.048 & 0.00027824 \\ 
                800 & 0.052 & 0.00026794 \\ 
                700 & 0.052 & 0.00023444 \\ 
                600 & 0.064 & 0.00024733 \\ 
                500 & 0.068 & 0.00021899 \\ \hline
            \end{tabular}
        \end{table}
    \end{description}

    \item 実験課題2において,それぞれの発振周波数におけるロゴスキーコイルの出力
    から$I_d$を算出し,表にして示せ.
    \begin{description}
        \item[] 算出した変位電流$I_d$を表2に示す.また,$\mu_0=0.000001257\,\si{[H/m]}$
        ,$N=211$,$l=2\pi\times 0.1095=0.688\,[\si{m}]$,
        $S=\pi\times 0.0100^2=0.000314\,[\si{m^2}]$である.
        \begin{table}[H]
            \centering
            \caption{実験課題2における変位電流$I_d$}
            \begin{tabular}{c|c|c}
                \hline
                $f\,[\si{kHz}]$ & $A\,[\si{V}]$ & $I_{dmax}\,[\si{A}]$ \\ \hline
                1000 & 0.105 & 0.13805 \\ 
                900 & 0.085 & 0.12418 \\ 
                800 & 0.080 & 0.13148 \\ 
                700 & 0.065 & 0.12209 \\ 
                600 & 0.055 & 0.12052 \\ 
                500 & 0.045 & 0.11833 \\ \hline
            \end{tabular}
        \end{table}
    \end{description}

\newpage

    \item 二本のプローブを用いて得られた$I_d$とロゴスキーコイルから得られた$I_d$の値を
    比較検討せよ.
    \begin{description}
        \item 二本のプローブを用いて測定した変位電流$I_d$は,誘電率$80.4(20\si{\celsius})$
        の誘電体(水)の内部の電束密度から生じる.その一方で,ロゴスキーコイルを用いて測定
        した変位電流$I_d$は,誘電率$1.00$の誘電体(空気)の内部の電束密度から生じる.
        $$
        \vec{i_d}=\frac{\partial \vec{D}}{\partial t}
        $$
        より,プローブを用いて測定した変位電流$I_d$のほうがロゴスキーコイルを用いて測
        定した変位電流$I_d$より$80.4$倍大きいと考えられる.また,水の誘電率は温度が低く
        なるにつれ上がり,$10\,\si{\celsius}$では$84.14$になる.このことから,
        水の誘電率の温度依存性によって$84.14/80.4 \simeq 1.05$倍されたと考え
        られる.さらに,実験課題1ではノイズが多く発生しており,$\pm 20\si{mV}$の読み取り誤差が発生する.
        読み取り誤差による倍率は,$(0.055+0.02)/0.055 \simeq 1.364$倍となる.
        同様に,微小区間$\Delta x$やロゴスキーコイルの断面積,円周の長さによる,読み取り誤差に
        よる影響も考えられる.
        
        
        \begin{table}[H]
            \centering
            \caption{実験課題1,2それぞれで求められた変位電流の比}
            \begin{tabular}{c|cc|c}
                \hline
                $f\,[\si{kHz}]$ & 変位電流(実験課題1)$[\si{A}]$ & 変位電流(実験課題2)$[\si{A}]$ & 比 \\ \hline
                1000 & 0.00028339 & 0.13805 & 487.1465605 \\ 
                900 & 0.00027824 & 0.12418 & 446.2884911 \\ 
                800 & 0.00026794 & 0.13148 & 490.7153997 \\ 
                700 & 0.00023444 & 0.12209 & 520.7591997 \\ 
                600 & 0.00024733 & 0.12052 & 487.3076539 \\ 
                500 & 0.00021899 & 0.11833 & 540.3642519 \\ \hline
            \end{tabular}
        \end{table}
    \end{description}
    
\newpage

    \item 実験課題2について,ロゴスキーコイルから得られた$I_d$と,セメント抵抗の両端の電圧波形から
    得られる$I_F$の位相の相対関係を示せ.
    \begin{description}
        \item[] 平行平板に印加する$V$の角周波数を$\omega$とし,$V$が$\sin \omega t$
        に比例する場合,オームの法則$V=I_F R$より以下の関係が成り立つ.
        $$
        I_F\propto\cos\omega t
        $$
        $|\vec{i_d}|=\frac{\epsilon}{\Delta x}|\frac{\partial}{\partial t}(\Delta V)|$
        となるので,$I_d$は以下の関係が成り立つ.
        $$
        I_d\propto\sin \omega t
        $$
        したがって,ロゴスキーコイルから得られた$I_d$と,セメント抵抗の両端の電圧波形か
        ら得られる$I_F$の位相は$\pi$ずれる.

    \end{description}

    \item コンデンサーを含む回路では,$I_F$と$I_d$が閉ループを作るために,$I_F+I_d$はどのよう
    な面を取っても,$I_F+I_d=一定$となる.この予測が正しいかどうか得られた実験
    データに基づいて判定せよ.
    \begin{description}
        \item[] 実験結果より算出した$I_F+I_d$とそのデータ処理を表4に示す.
        相対誤差率は$-10\,\%~16\,\%$となっているので,$「I_F+I_d=一定となる」$とは言えない.
        \begin{table}[H]
            \centering
            \caption{$I_F+I_d$とそのデータ処理}
            \begin{tabular}{c|c|ccc|c}
                \hline
                ~ & $f\,[\si{kHz}]$ & $I_d\,[\si{A}]$ & $I_F\,[\si{A}]$ & $I_F+I_d\,[\si{A}]$ & 相対誤差率\,[\%] \\ \hline
                ~ & 1000 & 0.1380544747 & 0.0425 & 0.1805544747 & 16.52985447 \\ 
                ~ & 900 & 0.1241759826 & 0.0375 & 0.1616759826 & 4.345675992 \\ 
                ~ & 800 & 0.1314804521 & 0.03 & 0.1614804521 & 4.219480651 \\ 
                ~ & 700 & 0.1220889913 & 0.025 & 0.1470889913 & -5.068767911 \\ 
                ~ & 600 & 0.1205237478 & 0.02 & 0.1405237478 & -9.305975924 \\ 
                ~ & 500 & 0.1183324069 & 0.02 & 0.1383324069 & -10.72026728 \\ \hline
                平均 & ~ & ~ & ~ & 0.1549426759 \\ \hline
            \end{tabular}
        \end{table}
    \end{description}
    
\end{enumerate}