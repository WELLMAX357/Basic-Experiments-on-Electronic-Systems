% !TEX root = main.tex

%%%%%%%%%%%%%%%%%%%%%%%%%%%%%%%%%%%%%%%%%%%%%%%%%%%%%%%%%%%%%%%%%%%%%%%%%%%%%%%%%%%%%%%%%%%%%%%%
\section{宿題}
%%%%%%%%%%%%%%%%%%%%%%%%%%%%%%%%%%%%%%%%%%%%%%%%%%%%%%%%%%%%%%%%%%%%%%%%%%%%%%%%%%%%%%%%%%%%%%%%

\begin{enumerate}
    \item $\nabla \times \vec{H}=(i+\frac{\partial \vec{D}}{\partial t})$の両辺の発散をとることで,
    この式が電荷保存則$\frac{\partial \rho}{\partial t}+\nabla\cdot i=0$を確かに満たしていることを示せ.
    \begin{description}
        \item[] 両辺の発散をとると,任意のベクトル$\vec{A}$に関して,$\mathrm{div}(\mathrm{rot} \vec{A})$となることから,
        $$
        \nabla \cdot(\nabla \times \vec{H})=\nabla \cdot\left(i+\frac{\partial \vec{D}}{\partial t}\right)=\nabla \cdot i+\frac{\partial(\nabla \cdot \vec{D})}{\partial t}=0
        $$
        となる.
        ここで,ガウスの法則より$\mathrm{div}\vec{D}=\rho$であるので,以下の式が成り立ち,題意は示された.
        $$
        \nabla \cdot i+\frac{\partial(\nabla \cdot \vec{D})}{\partial t}=\frac{\partial \rho}{\partial t}+\nabla \cdot i=0
        $$

    \end{description}
\end{enumerate}