\subsection*{問題10}
マイコンレーサーに搭載のマイクロプロセッサは16bit幅の命令長,演算ビット幅,目盛りアクセス空間を持つ.
各種マイクロプロセッサの命令長を調べよ.
\begin{itemize}
    \item 32bitマイコン
    \begin{itemize}
        \item TMP19A31CYFG(東芝デバイス\&ストレージ株式会社)
        \item SPC560B40L3(ST マイクロエレクトロニクス株式会社)
    \end{itemize}
    \item 16bitマイコン
    \begin{itemize}
        \item MSP430 マイコン(テキサス・インスツルメンツ社)
        \item MAXQ(R)RISC マイクロコントローラ(マキシム・ジャパン)
    \end{itemize}
    \item 8bitマイコン
    \begin{itemize}
        \item TMP89FH40NG(東芝デバイス\&ストレージ株式会社)
        \item 78K ファミリ(ルネサスエレクトロニクス株式会社)
    \end{itemize}
\end{itemize}

\subsection*{問題11}
図47をもとに,表21のポート名を産めよ.また,表内のGPIOとは何を表すか調べよ.
\begin{description}
    \item[] マイコンのピン配置を表5に示す.
    \item[] GPIOは,General Purpose Input/Outputの略で,汎用I/Oポートとも言われ,
    GPIOは,I/O のうち、デジタル信号に関するピンのことを指す.
\end{description}

% \begin{table}[H]
%     \centering
%     \caption{マイコンのピン配置}
%     \includegraphics[scale=0.5]{table5.pdf}
% \end{table}

\subsection*{問題12}
集積回路のパッケージQFP意外の集積回路のパッケージを調べ,なぜ様々なパッケージが必要なのか考察せよ.
\begin{description}
    \item[] 集積回路のパッケージには,DIP,SIP,Jリード,グリッドアレー,
    SO,SOT,TOなどがある.また,様々なパッケージがある理由として,基板ごとに
    素子に求められるサイズや形が異なることや,大量生産に対応する形や精密性を
    求める形などそれぞれ用途にあった長所や短所があるためだと考えられる.
\end{description}

\subsection*{問題13}
LED信号機も多数のLEDで信号のランプを構成しているが,このLEDは一定の周期で点滅している.
この周期を調べよ.交通事故などの記録のため,ドライブレコーダーを搭載する車が増えている.「LED信号機対応」と記載されている
ドライブレコーダーが販売されている.どのような対策で対応しているのか,対応していないと何が起きるのかを調べよ.
\begin{description}
    \item[] LED信号機の周波数は,西日本で$60\,\si{Hz}$,東日本で$50\,\si{Hz}$である.
    LED信号機に対応していないドライブレコーダーは,シャッタータイミングと信号の点滅タイミング
    が同期してしまい,信号が長く消えてしまう現象が発生する.「LED 信号機対応」と
    記載されているドライブレコーダーは,フレームレートを信号機の周期$50\,\si{Hz}/60\,\si{Hz}$
    に同期しないように,$27~29.5\,\si{fps}$ にずらすことで対応している.
\end{description}

\subsection*{問題14}
図60には,入力Input,出力$O_1,O_2$ともに,ダイオードが接続されている.このダイオードは保護ダイオードと呼ばれている.
この役割を調べよ.
\begin{description}
    \item[] 保護ダイオードは,インターフェースなど外部端子から侵入する異常電圧
    を吸収し,回路の誤動作防止およびデバイスを保護する.静電気・短時間のパルス電
    圧吸収・抑制対策に用いられる.
\end{description}